\section{Introducción}

La caracterización sistemática de los fenotipos clínicos es un pilar esencial de la investigación biomédica moderna, particularmente en el marco de la medicina genómica y de precisión, la cual busca integrar información clínica y datos moleculares de forma estandarizada para avanzar en la comprensión de las enfermedades humanas. Tal como destacan Köhler et al. en su revisión publicada en \textit{Nucleic Acids Research} (2021) \cite{kohler2021hpo}, la \textbf{Human Phenotype Ontology} (HPO) ha permitido estructurar y estandarizar la descripción de los rasgos clínicos observables, favoreciendo la interoperabilidad entre bases de datos genómicas y fenotípicas a nivel internacional.

Robinson y colaboradores, en su artículo \textit{American Journal of Human Genetics} (2008) \cite{robinson2008hpo}, ya recalcabal el uso de la HPO, no solo porque facilitara la anotación uniforme de enfermedades hereditarias, sino porque también potencia la capacidad de identificar correlaciones entre las variantes genéticas y sus manifestaciones clínicas. Siguiendo esta línea, Groza et al. (2015) \cite{groza2015hpo} consiguieron demostrar cómo la integración semántica de la HPO contribuye a unificar el análisis de enfermedades raras y comunes, mejorando la inferencia de mecanismos compartidos.

Esta estandarización fenotípica ha impulsado el desarrollo de enfoques de biología de sistemas aplicados al estudio de enfermedades humanas. Como ya explicó Barabási en su artículo de \textit{Nature Reviews Genetics} (2011) \cite{barabasi2011network}, las patologías pueden entenderse como alteraciones en las redes de interacción molecular, donde genes y proteínas actúan como nodos interconectados dentro de una arquitectura biológica compleja. Según esta perspectiva, el estudio de un fenotipo no queda limitado a la identificación de un único gen causante, sino que se extiende al análisis de redes completas funcionales que pueden conseguir revelar módulos biológicos implicados en la enfermedad.

Según este planteamiento, el fenotipo \textit{Abnormal drinking behaviour} (conducta anormal de ingesta de líquidos) se define como una alteración en los patrones fisiológicos normales de consumo de agua según la HPO \cite{kohler2021hpo}. Tal y como se describe en el \textit{Textbook of Medical Physology} de Guyton Hall (2021) \cite{guyton2021physiology}, este fenotipo generalmente incluye manifestaciones como la \textit{polidipsia} (ingesta excesiva de agua asociada con alteraciones endocrinas o neurológicas), la \textit{hipodipsia} (disminución del impulso o deseo de beber) y la \textit{adipsia} (ausencia completa de sensación de sed). Estas alteraciones reflejan disfunciones en los mecanismos homeostáticos encargados de regular el equilibrio hídrico en nuestro cuerpo, controlados mediante circuitos hipotalámicos, la secreción de vasopresina y la función renal.

El rol del sistema nervioso central en la regulación de la sed fue explicado por Verbalis en un estudio publicado en el \textbf{Journal of the American Society of Nephrology} (2007) \cite{verbalis2007brain}, en el cual se describe cómo el cerebro es capaz de detectar los cambios osmóticos que se producen y, como consecuencia, desencadenar respuestas neuroendocrinas adaptativas. Complementariamente, Antunes-Rodrigues et al. (2004) \cite{antunes2004neuroendocrine} mostraron en su revisión de \textit{Physiological Reviews} que el control neuroendocrino de la homeostásis hídrica depende en gran parte de la integración de múltiples señales hormonales y neuronales, con especial hincapié en la interdependencia entre cerebro, hipófisis y riñones.

\textbf{Objetivo e hipótesis}. En este trabajo se propone aplicar un enfoque de biología de sistemas para tratar de analizar los genes asociados al fenotipo \textit{Abnormal drinking behaviour}, tal como se encuentran descritos en la HPO. Mediante la construcción y el análisis de \textbf{redes de interacción génica}, se pretende identificar módulos funcionales y genes clave que se encuentren potencialmente implicados en la regulación de la homeostásis hídrica. Siguiendo este enfoque, a través del análisis topológico y funcional de las redes, podremos generar nuevas hipótesis sobre los mecanismos biológicos subyacentes a este fenotipo, logrando así aportar una visión global y reproducible de su arquitectura molecular. De este modo, el fin de nuestro proyecto es avanzar hacia una comprensión más profunda de cómo las alteraciones genéticas pueden traducirse en manifestaciones fenotípicas observables, contribuyendo al estudio de la variablidad fisiológica y patológica del comportamiento de ingesta de líquidos.
